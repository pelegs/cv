%%%%%%%%%%%%%%%%%
% This is an sample CV template created using altacv.cls
% (v1.7.4, 30 Jul 2025) written by LianTze Lim (liantze@gmail.com), based on the
% CV created by BusinessInsider at http://www.businessinsider.my/a-sample-resume-for-marissa-mayer-2016-7/?r=US&IR=T
%
%% It may be distributed and/or modified under the
%% conditions of the LaTeX Project Public License, either version 1.3
%% of this license or (at your option) any later version.
%% The latest version of this license is in
%%    http://www.latex-project.org/lppl.txt
%% and version 1.3 or later is part of all distributions of LaTeX
%% version 2003/12/01 or later.
%%%%%%%%%%%%%%%%

%% Use the "normalphoto" option if you want a normal photo instead of cropped to a circle
% \documentclass[10pt,a4paper,withhypeper,normalphoto]{altacv}

\documentclass[10pt,a4paper,withhyper]{altacv}
%% AltaCV uses the fontawesome5 and simpleicons packages.
%% See http://texdoc.net/pkg/fontawesome5 and http://texdoc.net/pkg/simpleicons for full list of symbols.

% Change the page layout if you need to
\geometry{left=1.25cm, right=1.25cm, top=1.25cm, bottom=1cm, columnsep=1.75cm}

% The paracol package lets you typeset columns of text in parallel
\usepackage{paracol}

% Change the colours if you want to
\definecolor{VividPurple}{HTML}{3E0097}
\definecolor{SlateGrey}{HTML}{2E2E2E}
\definecolor{LightGrey}{HTML}{666666}
\definecolor{xblue}{HTML}{4268BD}
\definecolor{xpurple}{HTML}{7F52B2}
\colorlet{accent}{xblue!90}
% \colorlet{name}{black}
\colorlet{tagline}{accent}
\colorlet{heading}{accent}
\colorlet{headingrule}{accent}
% \colorlet{subheading}{PastelRed}
\colorlet{emphasis}{SlateGrey}
\colorlet{body}{LightGrey}

% Change some fonts, if necessary
% \renewcommand{\namefont}{\Huge\rmfamily\bfseries}
% \renewcommand{\personalinfofont}{\footnotesize}
% \renewcommand{\cvsectionfont}{\LARGE\rmfamily\bfseries}
% \renewcommand{\cvsubsectionfont}{\large\bfseries}

% Change the bullets for itemize and rating marker
% for \cvskill if you want to
\renewcommand{\cvItemMarker}{{\small\textbullet}}
\renewcommand{\cvRatingMarker}{\faCircle}
% ...and the markers for the date/location for \cvevent
% \renewcommand{\cvDateMarker}{\faCalendar*[regular]}
% \renewcommand{\cvLocationMarker}{\faMapMarker*}


% If your CV/résumé is in a language other than English,
% then you probably want to change these so that when you
% copy-paste from the PDF or run pdftotext, the location
% and date marker icons for \cvevent will paste as correct
% translations. For example Spanish:
% \renewcommand{\locationname}{Ubicación}
% \renewcommand{\datename}{Fecha}


%% Use (and optionally edit if necessary) this .tex if you
%% want to use an author-year reference style like APA(6)
%% for your publication list
% \input{pubs-authoryear.cfg}

%% Use (and optionally edit if necessary) this .tex if you
%% want an originally numerical reference style like IEEE
%% for your publication list
\input{pubs-num.cfg}

%% sample.bib contains your publications
\addbibresource{sample.bib}

%% Defines the multilanguage (EN/DE) command
\input{multilang}

%% Contains secrets such as phone number
%% i.e. stuff that shouldn't be on a public git repo
%% hence they are loaded from a 'secrets' file that is in the .gitignore
\input{secrets}

\begin{document}
\name{Peleg Sapir}
\tagline{\multilang{Software Developer}{Softwareentwickler}}
% Cropped to square from https://en.wikipedia.org/wiki/Marissa_Mayer#/media/File:Marissa_Mayer_May_2014_(cropped).jpg, CC-BY 2.0
%% You can add multiple photos on the left or right
\photoR{3.35cm}{pelegs1}
% \photoL{2cm}{Yacht_High,Suitcase_High}
\personalinfo{%
	\email{pelegs@gmail.com}
	\phone{\myphonenum}
	\linkedin{pelegsap}
	\github{pelegs}
	\location{Darmstadt}
	\nationality{\multilang{German}{Deutsch}}
}

\makecvheader

%% Depending on your tastes, you may want to make fonts of itemize environments slightly smaller
\AtBeginEnvironment{itemize}{\small}

%% Set the left/right column width ratio to 6:4.
\columnratio{0.6}

% Start a 2-column paracol. Both the left and right columns will automatically
% break across pages if things get too long.
\begin{paracol}{2}
	\vspace{1.5em}
	\cvsection{\multilang{Experience}{Erfahrung}}

	\cvevent{\multilang{Software Developer}{Softwareentwickler}}{SureSecure GmbH}{February 2025 -- July 2025}{Düsseldorf, Germany}
	As a member of the Securiry Operation Center (SOC) I was responsible for developing, maintaining and integrating various components and new features into our workflow, with emphasis on our SIEM and SOAR platforms.\\
	% \textbf{Note}: the specific details are confidential due to the nature of the cybersecurity environment. Further information will be divulged per request only as allowed by the company.

	\divider

	\cvevent{Software Developer, Automation Engineer}{exocad GmbH}{November 2020 -- January 2025}{Darmstadt, Germany}
	I was tasked with designing and implementing automated processes for different teams in the company.

	Examples of successfull projects (\textbf{tools used}):

	\vspace{0.5em}

	\begin{itemize}
		\item Designed, implemented and deployed a system for extraction of data from 3D models provided by external manufactures. The data is collected into a database which is updated daily and provides simple search and statistics to internal users via SQL queries. This system is used daily by the company's integration team. (\textbf{Python}, \textbf{Sqlite}, \textbf{Bash})
		\item Created a conversion tool between two ticketing systems: \textit{trac} and \textit{Jira}. I then used the tool to migrate all of the existing tickets, including comments, change histories and all other attributes. (\textbf{Python with REST API}, \textbf{PostgreSQL})
		\item Wrote script for anonymization of client-generated case files as part of a pipeline for developing machine-learning models. (\textbf{Python}, \textbf{C\#)}
		\item Created a pipeline for generating official release notes from relevant Jira tickets. (\textbf{Python}, \textbf{Jenkins})
	\end{itemize}

	\vspace{0.5em}
	I also gave seminars in the company on various topics, such as \textit{linear algebra} and \textit{python}.

	\divider

	\cvevent{Software Developer}{MPI for Biophysical Chemistry}{2018 --  2019}{Göttingen, Germany}
	Developed firmware and software for a locally built automatic worm incubator. Publication: \href{https://www.tandfonline.com/doi/full/10.1080/01677063.2020.1776709}{DOI: 10.1080/01677063.2020.1776709}.

	\divider

	\cvevent{Student Assistant (HiWi)}{ZEDAT, Freie Universität Berlin}{2017}{Berlin, Germany}
	Was a member of the HPC cluster group. As part of the team I wrote scripts for different management purposes.

	\divider

	\cvevent{Software Developer, Research Assistant}{MPI for Dynamics and Self-Organization, Göttingen}{2014 -- 2016}{Göttingen, Germany}
	Developed software to track collectives of bacteria from videos.


	%\cvsection{A Day of My Life}

	%% Adapted from @Jake's answer from http://tex.stackexchange.com/a/82729/226
	%% \wheelchart{outer radius}{inner radius}{
	%% comma-separated list of value/text width/color/detail}
	%% Some ad-hoc tweaking to adjust the labels so that they don't overlap
	%\hspace*{-1em}  %% quick hack to move the wheelchart a bit left
	%\wheelchart{1.5cm}{0.5cm}{%
	%	10/13em/accent!30/Sleeping \& dreaming about work,
	%	25/9em/accent!60/Public resolving issues with Yahoo!\ investors,
	%	5/11em/accent!10/\footnotesize\\[1ex]New York \& San Francisco Ballet Jawbone board member,
	%	20/11em/accent!40/Spending time with family,
	%	5/8em/accent!20/\footnotesize Business development for Yahoo!\ after the Verizon acquisition,
	%	30/9em/accent/Showing Yahoo!\ \mbox{employees} that their work has meaning,
	%	5/8em/accent!20/Baking cupcakes
	%}

	%% use ONLY \newpage if you want to force a page break for
	%% ONLY the currentc column

	\switchcolumn

	\vspace{1.5em}

	\cvsection{Summary}
	Software developer with \textbf{over 10 years} of professional experience.

	\cvsection{Main Tools}

	\cvachievement{\faPython}{Python}{Been working with python almost daily for \textbf{over 10 years}, it is my main go-to langauge for fast prototyping and scripting}

	\divider

	\cvachievement{\faLinux}{Linux}{Avid Linux user for \textbf{over 20 years} and fully fluent in any Linux environment}

	\divider

	\cvachievement{\faGit*}{Git}{Many years of experience working with git, including \textbf{GitLab CI/CD} for deployment \& integration}

	\divider

	\cvachievement{\faDocker}{Docker}{Experienced in setting up containers}

	\cvsection{Other Tools}

	% Don't overuse these \cvtag boxes — they're just eye-candies and not essential. If something doesn't fit on a single line, it probably works better as part of an itemized list (probably inlined itemized list), or just as a comma-separated list of strengths.

	\cvtag{Golang}
	\cvtag{C/C++}\\
	\cvtag{Bash}
	\cvtag{SQL}
	\cvtag{Jenkins}
	\cvtag{AWS}
	\cvtag{BitBucket}
	\cvtag{NeoVim}
	\cvtag{\LaTeX\tikzmark{ltx}}
	\begin{tikzpicture}[remember picture, overlay]
		\fontsize{6}{6}\selectfont{}
		\draw[stealth-] ($(pic cs:ltx) + (7pt,2.5pt)$) to [out=0, in=180] ++(25pt,-10pt) node [right, text width=1.15cm] {what I used to typeset the cv \faIcon[regular]{smile-beam}};
	\end{tikzpicture}

	% \divider\smallskip

	\cvsection{Languages}
	\vspace{0.25em}
	\begingroup
	\setlength{\parskip}{-0.5em}
	\langskill{English}{Fully fluent in speech and writing}
	\langskill{German}{Fluent in daily conversations}
	\langskill{Hebrew}{Native speaker}
	\endgroup

	\cvsection{Academia}
	\begingroup
	\setlength{\parskip}{-0.2em}

	\cvGradInfo{M.Sc. Chemistry}{Freie Universität Berlin}{2019}{Berlin/Göttingen}

	\divider

	\cvGradInfo{B.Sc. Chemistry}{Tel Aviv university}{2012}{Tel Aviv-Jaffa}

	\divider

	\cvTeachInfo{Lecturer: Physics Simulations}{DHBW}{2024}{Mannheim}

	\divider

	\cvTeachInfo{Lecturer: Maths for Biochemistry}{Georg-August-Universität}{2020 -- 2021}{Göttingen}

	\endgroup

\end{paracol}

\end{document}
